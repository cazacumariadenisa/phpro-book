\fancyfoot[CO,C] {\thepage}
\phantomsection
\addcontentsline{toc}{chapter}{Introducere}
\chapter*{Introducere}

\begin{chapsummary}

Acest capitol conține informații foarte importante despre cum să studiezi, unde
să ceri ajutor, cum să raportezi greșeli și în general cum să profiți la maxim
de materialul prezentat. Recomand citirea sa cu atenție.

\end{chapsummary}

În ultimii ani, utilizarea Internetului a crescut rapid. Numărul de dispozitive
conectate la Internet crește în mod exponențial de la an la an, iar
web-ul\footnote
{World Wide Web} a devenit scena principală. Web-ul nu mai este static demult,
avem rețele sociale,\footnote{hi5, facebook, lastFM, delicious}
feed-uri,\footnote{RSS, Atom} bloguri și microbloguri,\footnote{twitter}
ș.a.m.d. Observăm că mai toate activitățile noastre informaționale s-au mutat 
pe web.

Însă aceste activități devin din ce în ce mai complexe, la fel ca și
aplicațiile\footnote{Termenul \glqq aplicație\grqq nu este tocmai corect, în sensul
tradițional, însă autorul consideră că limba este ceva maleabil, care se
schimbă în timp în funcție de nevoi.} care le susțin.
Iar aceste aplicații trebuie dezvoltate de cineva -- de programatori.

PHP este unul dintre cele mai folosite limbaje pentru crearea de aplicații
web dinamice. Succesul său se datorează în special simplității sale, însă
acest lucru e cu două tăișuri: pe de o parte este ușor accesibil, pe de
cealaltă parte poți face foarte ușor greșeli majore.

\phantomsection
\addcontentsline{toc}{section}{Scopul acestei cărți}
\section*{Scopul acestei cărți}

Pentru a scrie aplicații PHP bune\footnote{performante, sigure, complexe, mentenabile}
este cerut simțul critic al programatorului,
însă există o mare problemă: începătorul care alege PHP ca
primul său limbaj de programare nu are un simț critic dezvoltat sau gândirea analitică
necesare în programare. În combinație cu accesibilitatea \textit{aparentă}
a limbajului PHP, acest lucru se dovedește fatal pe \textit{termen lung}.

Scopul acestei cărți este să-ți dezvolte atât gândirea autonomă, productivă,
critică, cât și capacitatea de analiză și sinteză, capacități atât de vitale
în programare.
%-----------------------------------------------------------------------------

Lucrarea de față nu este nici pe departe completă -- nici nu vrea să fie.
Tot ceea ce vrea este să-ți ofere o fundație bună de start. Acest lucru se 
face punând accent pe \textit{terminologie} și pe explicarea
\textit{modului de funcționare} a noțiunilor și tehnologiilor prezentate.

După cum probabil intuiești deja, scopul acestei cărți este să
te susțină să devii bun în perspectivă, fiind orientată spre
viitor, pe termen lung.

\phantomsection
\addcontentsline{toc}{section}{De ce am nevoie? Premize}
\section*{De ce am nevoie? Premize}
Această lucrare pleacă de la premiza că știi deja (X)HTML,\footnote{În
ciuda credințelor populare, HTML nu este un limbaj de programare.} eventual și CSS, dar
acesta din urmă nu este necesar pentru înțelegerea lucrurilor prezentate
sau pentru învățarea PHP. Ar trebui
studiat oricum, căci fără el nu este posibil \textsl{design}-ul de
\textsl{website}-uri aspectuoase.
Acolo unde va fi nevoie, vor fi prezentate și noțiunile JavaScript\footnote{Un
limbaj de programare pentru programarea clientului, a \textsl{browser}-ului,
în contrast cu PHP, cu care se programează \glqq serverul\grqq.
În ciuda credințelor din folclor, JavaScript și Java sunt limbaje complet
diferite și de sine stătătoare.} necesare.

Un lucru important de care ai nevoie este răbdare. Citește cu
atenție și încearcă să înțelegi tot, căci informația este comprimată
și uneori pare să nu aibă nicio aplicabilitate practică, însă e doar o
iluzie -- tot ce scrie în acest ghid scurt este important. Nu uita că
\textit{mă rezum doar la fundație, la cunoștințele de bază}.

Așa cum spune și coperta acestei cărți, plec de la premiza că
\textit{vrei să devii profesionist în PHP}. Dacă \textit{nu} asta este intenția ta,
atunci lucrarea de față nu este potrivită pentru tine.
În particular, în timp ce urmezi această carte pentru prima oară, nu o face
pentru a-ți rezolva problema ta imediată de care tocmai te-ai lovit, ci
încearcă să înțelegi noțiunile expuse și să rezolvi exercițiile prezentate.
Vei vedea că asta îți salvează mult timp și frustrare pe \textit{termen lung},
și că rezolvarea eventualei probleme imediate de care te-ai lovit este de fapt
marginală carierei tale de \textit{programator profesionist}.

\attention{Ca un viitor profesionist ce ești, citește cu atenție acest material, și
încearcă să înțelegi nu numai conceptele de care te lovești, ci și
implicațiile lor. Analizează-le, atât pe el însele, cât și în relație
cu celelalte concepte introduse. Cu cât sintetizezi mai mult atunci
când întâlnești ceva nou, cu atât vei ajunge să \textit{jonglezi} cu noțiunile
învățate mai rapid, lucru care îți va permite să fii inovativ.

De exemplu, în primul capitol vei învăța despre rețelistică. Întreabă-te
pe parcursul întregii cărți ce efecte au limitările HTTP asupra posibilităților
sau asupra securității.}

Un alt lucru de care ai nevoie este stăpânirea limbii române. Experiențele
noastre cu cursanții ne-au învățat că mulți dintre cei ce aspiră a fi
programatori nu îndeplinesc această premiză. Unii dintre ei au avut dificultăți
nu pentru că nu ar ști să vorbească, ci pentru că nu au realizat intensitatea
cu care punem accent pe acest aspect. Nu te teme, prima fază
%TODO spune despre stagiile de experiență, aici: novice
de tutelare are exact acest scop: să te aducă pe linia de plutire.

Și nu în ultimul rând, ai nevoie de stăpânirea relativ bună a limbii engleze.
Fără ea oricum nu ai avea succes în programare -- de multe ori va trebui
să citești documentații în engleză, ba mai bine, să îți documentezi aplicațiile
în engleză. Apropo de documentație, toate proiectele de succes au o documentație
bună în engleză. Nu te impacienta dacă nu te simți pregătit să scrii ceva în
engleză, până la capitolul 3 inclusiv vei avea șansa să îți îmbunătățești engleza
doar citind.

\phantomsection
\addcontentsline{toc}{section}{Convenții folosite}
\section*{Convenții folosite}
Pentru a-ți face lectura cât mai plăcută, cartea de față respectă anumite
convenții, atât de natură tipografică, cât și inerente comunității din jurul
lucrării.

În primul rând, pe prima pagină se află un \textsl{link} către pagina de
start a proiectului. Această pagină va fi numită mereu \textit{pagina \phpro}.
Următoarea secțiune îți va descrie despre ce este vorba în detaliu.

În al doilea rând, când spun wikipedia, mă refer la versiunea originală
în engleză a site-ului \url{http://en.wikipedia.org/}. În carte nu voi face referire
la niciun \textsl{site} în română.


{\glqq}Pagina PHP{\grqq} și {\glqq}manualul PHP{\grqq} se referă la paginile oficiale \url{http://php.net/}
și respectiv \url{http://www.php.net/manual/en/}. Demn de menționat este
că nu voi folosi decât \textit{site}-urile oficiale ale produselor despre
care vorbesc.

\attention{Urmarea link-urilor, în special cele către wikipedia și către
manualul PHP, \textit{nu} este
opțională. Informațiile prezentate în acele pagini \textit{fac parte} din cartea
de față.}

Cuvintele importante sunt scrise \textit{cursiv}, termenii și noțiunile
importante sunt scrise \textsl{înclinat}, iar cuvintele care fac referire
la nume de funcții, comenzi sau instrucțiuni care trebuie introduse
într-un fișier sau ca comandă,
sau cuvinte cheie specifice unui anumit limbaj
sunt scrise cu font \texttt{neproporțional}\footnote{\href{http://en.wikipedia.org/wiki/Monospaced_font}{monospace}}.

Apăsările de taste sunt scrise în chenare, astfel \keystroke{ENTER} înseamnă
să apeși enter o dată, iar \keystroke{CTRL+F} înseamnă să apeși tasta \keystroke{CTRL}
și în timp ce o ții apăsată, să apeși \keystroke{F}.

Codul sursă, de obicei în PHP, va arăta în felul următor:

\vskip 3em
\begin{lstlisting}[caption={Convenție listare}]
<?php
phpinfo();
\end{lstlisting}

Numerele de pe marginea stângă reprezintă numerele liniilor de cod corespunzătoare
și nu trebuie scrise. Ele deservesc unei mai ușoare identificări în explicațiile
din text.

Codul sursă va conține și caracterul ’. Acesta trebuie văzute ca
apostrof. Altfel spus, copierea și lipirea codului sursă prezentat
nu va funcționa pur și simplu. Acesta trebuie scris de tine însuți, deoarece
caracterul respectiv este (intenționat) greșit. Procedez astfel pentru
a împiedica copierea codului sursă. Acesta trebuie înțeles.

Atunci când introduc termeni noi, încerc să ofer și traducerea în
engleză, în paranteză.
Exemplu: \begin{quote}
         	Un astfel de atac se numește man in the middle, deoarece atacatorul se află la mijloc, între
cele două capete (en. \textsl{endpoints}).
         \end{quote}
iar la sfârșitul cărții poți găsi o referință a tuturor acestor termeni.

Există trei feluri în care marchez paragrafe:

\attention{Paragrafele care te atenționează asupra unui lucru important
sunt marcate ca atare, ca cel de față.}

\good{În unele locuri vreau să-ți atrag atenția asupra unei practici de programare bune.}

\bad{În același timp, câteodată îți atrag atenția asupra unor lucruri care, deși sunt posibile, nu ar trebui făcute.}

Exercițiile sunt marcate cu un creion, iar numărul de steluțe reprezintă dificultatea lor, între 0 (nicio steluță)
și 3 (trei steluțe). Exemplu:
\begin{Exercise*}[title={Exercițiu de dificultate 1},difficulty=1]
	Eu sunt un enunț.
\end{Exercise*}

Exercițiile de dificultate zero necesită doar înțelegerea exemplelor și
explicațiilor imediat anterioare și mici modificări sau adăugiri.
Cu cât solicitarea inteligenței și a capacității de sinteză a cursantului
crește, cu atât crește și numărul steluțelor.

La evaluarea dificultății exercițiilor procedez în
felul următor: în primul rând, plec de la premiza că
cursantul știe toate noțiunile din capitolul anterior,
chiar dacă nu le-a sintetizat pe toate. În același timp,
plec de la premiza că restul capitolelor trecute
au fost bine sintetizate.

\attention{Dacă trebuie să sari cu mai mult de un
capitol înapoi pentru a revizui ceva, atunci este
un indiciu că nu ai trecut prin toate stadiile
de studiu în mod consecvent. Secțiunea următoare
îți va explica care sunt aceste stadii.}

%La începutul unui
%capitol plec de la premiza că cursantul
%a înțeles deja noțiunile capitolului anterior
%jonglează deja fără probleme
%cu noțiunile acoperite de capitolele anterioare. Altfel spus, nu te poți
%aștepta ca exercițiile dintr-un nou capitol să fie ușoare, dacă
%nu ai trecut prin toate stadiile unui studiu corect. Următoarea
%secțiune îți va prezenta aceste stadii și care sunt beneficiile lor.

Dacă observi încălcări ale acestor convenții, te rog să le
raportezi pe pagina de greșeli a \phpro.
\phantomsection
\addcontentsline{toc}{section}{Cum să înveți eficient programare}
\section*{Cum să înveți eficient programare}

%TODO at chapter 6: remove this
Momentan cartea de față nu acoperă încă materia așa cum aș vrea -- nu este completă.
Însă subiectele abordate sunt acoperite complet, cel puțin la nivel conceptual.

Cartea în sine nu este gândită pentru a fi folosită singură, ci
în paralel cu comunitatea \phpro. În particular, unele exerciții
chiar nu sunt gândite pentru a fi rezolvate de cititor singur,
ci cu susținerea tutorilor de pe \phpro.

Pe {\phpro} găsești și ajutor sub formă de idei și indicii pentru rezolvarea
exercițiilor.

Învață \textit{terminologia}, înțelege-o și folosește-o. Dacă setul de scule de programare folosite
este lancea ta de programator, atunci terminologia este vârful lancei.
Care este diferența dintre un toiag tocit, și o lance fără vârf?
Exact, nici una. Nu te apuca să foloșesti termeni pe care nu-i înțelegi,
ci documentează-te înainte. Cu o lance ascuțită:
\begin{itemize}
	\item te vei putea înțelege mai ușor cu alți programatori; tu îi vei înțelege pe ei, și ei pe tine
	\item pe măsură ce termenii înțeleși de tine devin mai complecși,
		  vei putea acumula cunoștințe din ce în ce mai complexe bazate pe cele anterioare,
		  în ritm exponential. La început ți se va pare frustrant, însă dacă vrei să devii bun,
		  oricum va trebui să înveți termenii odată și-odată. Deci de ce să nu faci totul ca
		  la carte de la bun început?
	\item un programator profesionist știe mai mult de un singur limbaj de programare; ai fi
		  uimit dacă ai afla câți termeni și câte concepte sunt comune multor limbaje. Dacă
		  știi terminologia, chiar dacă ai învățat-o în (cu) PHP, vei putea trece la un nou
		  limbaj cu mult mai puține eforturi. Primul limbaj (învățat corect) este cel mai greu,
		  apoi ți se va pare floare la ureche
\end{itemize}


Urmează \href{http://en.wikipedia.org/wiki/Hyperlink}{link}-urile în timp ce
studiezi;
acestă carte nu este și nu va fi niciodată {\glqq}completă{\grqq} -- se pleacă de la premiza
că citești și înțelegi ce se află la acele link-uri \textit{înainte} de a trece
mai departe.

Notele de subsol sunt importante; dacă acestea introduc termeni neexplicați
anterior sau în imediata vecinătate, atunci trebuie reținute și făcute
legături atunci când termenii respectivi sunt introduși pentru prima oară.

Plec de la premiza că cititorul meu are un anumit nivel de inteligență.
Asta nu înseamnă că nu iau în serios orice nelămurire. Însă mă aștept
ca noțiunile prezentate să fie citite cel puțin, și apoi înțelese.
Nu are rost să citești o carte dacă ... nu o citești cu trup și suflet.
Atunci când ai o problemă, ia-o gradual, netrecând la următorul stadiu
până nu îl îndeplinești pe cel anterior.

\attention{Stadiile\footnotemark sunt: citire, înțelegere, sinteză, imaginație (jonglarea
cu noțiunile), inovație.}
\footnotetext{Noțiunea de
\textit{stadiu de învățare} este extinderea autorului a sistemului japonez
shu-ha-ri (en. \textit{retain-detach-transcend}, jp. \jptext{守 破 離}). Detalii pe
\url{http://www.makigami.info/cms/japanese-learning-system-japan-36}.}

A sintetiza înseamnă a face legături cu toate celelalte noțiuni
deja învățate. De exemplu vei învăța ce înseamnă un \textsl{array}, iar peste
câteva capitole vei face cunoștință cu obiecte. Dacă vei sintetiza
cum trebuie, îți vei da seama singur că este foarte posibil
să ai un array de obiecte.

A jongla cu noțiunile are ca efect practic faptul că cititorul știe
să pună în practică și să combine lucrurile învățate de ca și cum
acele noțiuni ar fi fost inventate de el.

\attention{Îți poți ușura procesul de sinteză asimilând terminologia
încă din momentul introducerii ei.}

Această sinteză e foarte importantă, și de fapt, o faci de când erai
copil. De exemplu, ai văzut-o pe mama ta tăind legumele cu cuțitul.
Mai târziu, la joacă, ai avut nevoie să tai o ață, și nu aveai decât
un cuțit în apropiere. Ți-ai dat seama că poți tăia ața cu cuțitul,
deși nu este o legumă. Altfel spus, ai sintetizat scopul uneltei
{\glqq}cuțit{\grqq}: să taie ceva.

Lucrarea de față explică foarte bine noțiunile, de la zero, însă
sinteza îți este lăsată ție. Motivația mea de a proceda așa este următoarea:
după cum sugerează subtitlul cărții -- \textit{Pentru începătorii în programare și în PHP care vor să devină profesioniști} -- scopul meu e să te îndrum pe calea profesionalismului.
Pe de cealaltă parte, sunt un darwinist convins, și dacă nu reușești
nici să devii profesionist, nici să vezi utilitatea acestei cărți, atunci
e mai bine așa. Ultimul lucru pe care îl vreau este să te susțin
să devii ceva în care nu ai avea succes.

%TODO uncomment at chap 6
Capacitatea de sinteză pe care o vei fi având la sfârșitul cărții
mai are încă un efect pozitiv asupra viitorului profesionist din tine:
în programare, vei fi confruntat cu nevoia de a reutiliza codul pe care-l scrii, astfel
încât să nu fii nevoit să rescrii același cod iar și iar, doar pentru
că trebuie să-l personalizezi puțin. Însă pentru a putea face
codul atât de flexibil încât să-l poți adapta cu ușurință, trebuie
să prevezi cazuri {\glqq}imprevizibile{\grqq}; altfel spus, să te gândești
la imposibil.

\good{Nu copia pur și simplu exemplele din carte, pentru că riști
să te trezești la un moment dat că nu ești în stare
să scrii ceva de unul singur. În schimb citește cu atenție codul
și explicațiile de dinaintea și după el, apoi \textit{închide cartea} și scrie totul
din minte, argumentându-ți (pe baza explicațiilor pe care le-ai citit)
de ce faci un lucru într-un anumit fel, sau de ce îl faci de fapt.}

Știu că este mai ușor să copiezi, dar vor veni vremuri când va
trebui să inventezi singur un script. Deci obișnuiește-te de
pe acum să scrii singur, și de ce nu, să faci greșeli. Atunci
când faci o greșeală și PHP îți spune asta, citește cu atenție
mesajul de eroare, apoi corectează-ți codul, și ține minte
pentru fiecare fel de greșeală ce eroare generează, pentru ca
în viitor să poți identifica mai rapid greșelile pe care le faci
pe baza mesajelor de eroare pe care ți le arată PHP.

\attention{Această \textit{putere de imaginație}, în combinație
cu \textit{capacitatea ta de analiză și sinteză}, și pe o fundație solidă
a \textit{înțelegerii conceptelor și termenilor} cu care intri în contact,
sunt cheia succesului garantat.}

\phantomsection
\addcontentsline{toc}{subsection}{Comunitatea}
\subsection*{Comunitatea}
\textit{Dezvoltare web cu PHP -- Pentru începătorii în programare și în PHP care vor să devină profesioniști}
nu este pur și simplu o carte, ci o comunitate și o serie de servicii pe care
această comunitate le oferă. Cartea de față constituie doar scheletul, fundația
studiului. Pentru a beneficia deci de aceste servicii, cititorul cărții
trebuie să fie și cursant în cadrul comunității.

Pagina {\phpro} este pagina de start a comunității. Printre serviciile oferite se numără:
\begin{itemize}
	\item verificarea soluțiilor exercițiilor și oferirea de indicii acolo unde cursantul s-a blocat, individual,
pentru fiecare cursant în parte, exact acolo unde are nevoie
	\item clarificarea nelămuririlor pe care cursantul le are în urma citirii explicațiilor
	\item articole care întregesc conceptele prezentate în carte; excursuri
	\item garanția că cursanții\footnote{În special cei care au reușit
să ofere soluții la primele trei exerciții din capitolul 2, eventual cu susținerea
tutorilor} au într-adevăr potențialul de a deveni profesioniști
	\item servicii care sunt folosite în viața reală a unui programator
\end{itemize}

Comunitatea {\phpro} nu este un loc unde poți primi ajutor la
problemele de care te-ai lovit pe cont propriu. Altfel spus, comunitatea
noastră este strict una de studiu.

\phantomsection
\addcontentsline{toc}{subsection}{Exercițiile}
\subsection*{Exercițiile}

Exercițiile sunt parte integrantă a studiului. Scopul exercițiilor nu
este numai de a te testa, ci și de a te învăța lucruri noi. De fapt,
unele exerciții au menirea exclusivă de a te învăța ceva.

Indiferent de menirea fiecărui exercițiu, poți apela la comunitatea
{\phpro} pentru susținere, sfaturi și indicii la exerciții. În fapt,
chiar va trebui să o faci la unele exerciții -- vei avea nevoie de asta.

Desprinzăndu-te de comunitatea \phpro, riști să studiezi ceva de unul
singur și să ai impresia că ai înțeles totul corect, însă lucrurile învățate
se pot așterne greșit în mintea ta, și la un moment dat te vei lovi
tu însuți de probleme din cauza asta.

Având însă permanent, la fiecare exercițiu, un tutore lângă tine care te
îndrumă, șansele ca un concept de programare să fie înțeles și aplicat
greșit scad considerabil.

Unele exerciții vor fi direct legate de comunitate și de serviciile pe care
aceasta le oferă. În capitolul patru de exemplu, exercițiile îți vor 
cere să formezi echipe cu alți cursanți, și să concurezi împotriva altor echipe, folosind
scule de programare așa cum sunt folosite în viața reală a unui programator,
precum un \textsl{bug tracker} sau un \textsl{revision control system}.

Însă pentru a primi acces la aceste servicii pe care comunitatea
{\phpro} le oferă gratis, trebuie să rezolvi toate exercițiile anterioare
sub tutela comunității, dovedind astfel că ai potențialul unui
programator bun.

\phantomsection
\addcontentsline{toc}{section}{Cum pot ajuta?}
\section*{Cum pot ajuta?}
Atât programatorii experimentați, cât și începătorii, pot ajuta,
iar ajutorul lor este apreciat în egală măsură.

Punctul de întâlnire pentru toți este \phpro, unde poți
găsi îndrumare despre ce poți face, sau unde poți raporta
ce ai de raportat.

De la cititorii avansați mă aștept la critică constructivă, sfaturi sau idei.
\textsl{Feedback}-ul mă bucură, însă vreau să atrag atenția asupra unui lucru:
există situații în care, atunci când trebuie să explici ceva, trebuie să
faci compromisuri între corectitudinea tehnică și ușurința cu care noțiunile
pot fi acumulate de cititor (en. \textsl{the learning curve}), iar cu compromisurile
suntem obișnuiți din programare. Așa se face că pe alocuri ofer explicații
nu tocmai corecte, care sunt corectate apoi. Asigură-te că ai citit tot conținutul
relevant (și mai ales notele de subsol) înainte de a raporta o greșeală --
cel mai probabil explicațiile sau definițiile sunt reluate și șlefuite undeva.

În privința calității cărții, există trei mari probleme:
\begin{itemize}
\item Nu cred în cacofonii. Consider că propria imaginație e singura vinovată
dacă {\glqq}vezi{\grqq} alte lucruri când citești. Ca atare, refuz să le corectez.
Ba mai rău, corectarea lor prin folosirea virgulei sau reformulări mai mult
ar îngreuna inteligibilitatea.
\item Folosesc xenisme. Resursele în limba engleză sunt cele mai acurate și
cele mai actuale, din acest motiv nu încerc să evit folosirea lor.
Asta îți va permite, pe \textit{termen lung}, să te poți ajuta singur. Pentru a
articula un xenism pun cratimă, și apoi particula specifică. De exemplu
\textsl{web}-ul; însă: \textit{Internetul} -- deoarece cuvântul internet există în
limba română.
\item Este foarte posibil să întâlnești formulări ciudate, cu ordinea
cuvintelor inversată, și altfel de greșeli similare. Cauza acestui lucru este că
90\% din timp vorbesc germana, lucru care-și lasă amprenta.
Corecturile sunt binevenite.
\end{itemize}

\phantomsection
\addcontentsline{toc}{section}{O privire de ansamblu a capitolelor}
\section*{O privire de ansamblu a capitolelor}
%TODO aici vine roadmap-ul

\begin{enumerate}
\item Rețelistică -- noțiunile de rețelistică sunt necesare pentru a înțelege mai
ușor apoi lucruri legate de securitate, optimizare sau servicii web
\item Controlul fluxului de execuție și de date -- te învață constructele pentru
controlul informațiilor în cadrul aplicației
\item Reutilizarea și modularizarea codului -- împărțirea codului în funcții și
fișiere, separarea logicii aplicației de vizualizare
\item Baze de date și lucrul în echipă -- cum să lucrezi în echipă, de la anumite
reguli de comunicare, până la mailing lists și git; documentarea proiectului;
debugging și profiling pentru a lua cele mai bune decizii; database design până
la și inclusiv 3rd normal form
\item Securitatea aplicațiilor web -- XSS, sql injection, CSRF
\item Programare orientată pe obiecte -- OOP, concepte generale, câteva patterns
(helper, strategy, factory, singleton), test-driven development, SPL
\item ajax, json, servicii (REST, SOAP, XML-RPC), XML, PDO și alte delicii, și ca
{\glqq}ultima frontiera{\grqq}: php internals.
\end{enumerate}

Începând cu capitolul 4 proiectele se vor realiza în echipe, iar tutorii vor avea
doar rolul de consilieri. Proiectele rezultate astfel vor aparține respectivilor
programatori, și vor putea fi folosite pentru portofoliile cursanților.

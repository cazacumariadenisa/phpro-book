\introduction{Introducere}

\begin{chapsummary}

Acest capitol conține informații foarte importante despre cum să studiezi, unde
să ceri ajutor, cum să raportezi greșeli și în general cum să profiți la maxim
de materialul prezentat. Este recomandată citirea sa cu atenție.

\end{chapsummary}

În ultimii ani, utilizarea Internetului a crescut rapid. Numărul de dispozitive
conectate la Internet crește în mod exponențial de la an la an, iar
\index{term}{World Wide Web}
\textsl{World Wide Web}\footnote{Abreviat
    \textsl{WWW}\index{abv}{WWW|see{World Wide Web}} sau pe scurt
    \textsl{web}\index{term}{web}\index{en-ro}{web}}
a devenit scena principală. \textit{Web}-ul nu mai este static demult, avem
rețele sociale\footnote{google+, facebook, lastFM, delicious},
\textsl{feed}-uri\footnote{RSS,\index{abv}{RSS|see{Rich Site Summary}}%
    \index{term}{RSS}%
    \index{term}{Rich Site Summary}%
    \index{en-ro}{Rich Site Summary}
Atom},
bloguri și microbloguri\footnote{twitter}, ș.a.m.d. Observăm cum mai
toate activitățile noastre informaționale s-au mutat pe \textit{web}.

Însă aceste activități devin din ce în ce mai complexe, la fel ca și
aplicațiile\footnote{Termenul ``aplicație'' nu este tocmai corect, în sensul
tradițional, însă autorul consideră că limba este ceva maleabil, care se
schimbă în timp în funcție de nevoi.} care le susțin.  Iar aceste aplicații
trebuie dezvoltate de cineva -- de programatori.

PHP este unul dintre cele mai folosite limbaje pentru crearea de aplicații
\textit{web} dinamice. Succesul său se datorează în special simplității sale,
însă acest lucru are două tăișuri: pe de o parte este ușor accesibil, pe de
cealaltă parte poți face foarte ușor greșeli majore.

% {{{ Section: Scopul acestei cărți
\section*{Scopul acestei cărți}
\phantomsection
\addcontentsline{toc}{section}{Scopul acestei cărți}

Pentru a scrie aplicații PHP bune\footnote{performante, sigure, complexe,
mentenabile} este cerut simțul critic al programatorului, însă există o mare
problemă: începătorul care alege PHP ca primul său limbaj de programare nu are
un simț critic dezvoltat sau gândirea analitică necesare în programare. În
combinație cu accesibilitatea \textit{aparentă} a limbajului PHP, acest lucru
se dovedește fatal pe \textit{termen lung}.

Scopul acestei cărți este să dezvolte cititorului atât gândirea autonomă,
productivă, critică, cât și capacitatea de analiză și sinteză, capacități atât
de vitale în programare.

Lucrarea de față nu este nici pe departe completă -- nici nu vrea să fie.  Tot
ceea ce vrea este să-i ofere cititorului o fundație bună de start. Acest lucru
se face punând accent pe \textit{terminologie} și pe explicarea \textit{modului
de funcționare} al noțiunilor și tehnologiilor prezentate.

După cum probabil intuiești deja, scopul acestei cărți este de a susține
cursantul să devină bun în perspectivă, fiind orientată spre viitor, pe termen
lung, și nu pe satisfacții de moment.

%TODO separa referirea la curs de cartea finala
Pentru a putea atinge acest scop într-un mod eficient, autorul scrie și
îmbunătățește cartea observând cu atenție începători în programare pe care îi
îndrumă în cadrul unui program de tutelare oferit gratuit de comunitate.
Această comunitate este formată din \textit{tutori} și cursanți, cursanții mai
avansați încearcând și ei la rândul lor să îi ajute pe cei mai puțin avansați.
În acest fel se ajunge la un efect de bulgăre de zăpadă, din care toată lumea
are de câștigat pe termen mediu și lung.

% }}}
% {{{ Section: De ce am nevoie? Premize
\section*{De ce am nevoie? Premize}
\phantomsection
\addcontentsline{toc}{section}{De ce am nevoie? Premize}

%TODO carte vs. curs
Această lucrare pleacă de la premiza că cititorul știe deja
(X)HTML\footnote{În ciuda credințelor populare, HTML nu este un limbaj de
programare. Poți să argumentezi de ce?}, eventual și
\textsl{CSS}\footnote{Cascading Style Sheet}
    \index{abv}{CSS|see{Cascading Style Sheet}}%
    \index{term}{Cascading Style Sheet}%
    \index{important}{Cascading Style Sheet}%
    \index{en-ro}{Cascading Style Sheet}%
dar acesta din urmă nu este necesar pentru
înțelegerea lucrurilor prezentate sau pentru învățarea PHP. Ar trebui studiat
oricum, căci fără el nu este posibil
\textsl{design}-ul
    \index{engl}{web design}%
    \index{term}{web design}%
de \textit{website}-uri aspectuoase.

Acolo unde va fi nevoie, vor fi prezentate și noțiunile
\textsl{JavaScript}\footnote{Un
limbaj de programare a clientului, a \textsl{browser}-ului, în contrast cu PHP,
cu care se programează ``serverul''.  În ciuda credințelor din folclor,
JavaScript și Java sunt limbaje complet diferite și de sine stătătoare.}
    \index{important}{JavaScript}%
necesare. Acest lucru se va întâmpla totuși într-un moment din evoluția
cititorului ca programator în care se consideră că acesta este extrem de
independent și face față studiului individual al unui nou limbaj. Vor fi expuse
în primul rând termenii și noțiunile specifice programării în
\textit{JavaScript} pentru web.

Un lucru important de care cititorul are nevoie este răbdare. Acesta trebuie să
citească cu atenție și să încerce să înțeleagă tot, căci informația este
comprimată și uneori pare să nu aibă nicio aplicabilitate practică, însă e doar
o iluzie -- tot ce scrie în acest ghid scurt este important.

\attention{Nu uita că \textit{acest curs se rezumă doar la fundație, la
cunoștințele de bază}.}

Așa cum spune și coperta acestei cărți, se pleacă de la premiza că
\textit{cititorul vrea să devină profesionist în PHP}. Dacă \textit{nu} aceasta
este intenția sa, atunci lucrarea de față nu este potrivită.

Cartea de față nu este potrivită nici pentru cititorii începători în programare
care au o problemă concretă de programare și care caută o soluție la aceasta.
Un începător ar trebui să își canalizeze energia încercând să înțeleagă
noțiunile expuse. În acest fel se economisește mai mult timp și frustrare
\textit{pe termen lung}. După parcurgerea cărții, cititorul își va da seama că
rezolvarea problemei sale imediate este în fapt marginală carierei sale de
\textit{programator profesionist}.

\attention{Ca un viitor profesionist, citește cu atenție acest material, și
    încearcă să înțelegi nu numai conceptele de care te lovești, ci și
    implicațiile lor. Analizează-le, atât pe ele însele, cât și în relație cu
    celelalte concepte introduse. Cu cât sintetizezi mai mult atunci când
    întâlnești ceva nou, cu atât vei ajunge să \textit{jonglezi} cu noțiunile
    învățate mai rapid, lucru care îți va permite să fii inovativ.

    \vspace{1em}
    De exemplu, în primul capitol vei învăța despre rețelistică. Întreabă-te pe
    parcursul întregii cărți ce efecte au limitările HTTP asupra posibilităților
    sau asupra securității.}

Un alt lucru de care este nevoie este stăpânirea limbii române. Experiențele
autorului cu cursanții au arătat că mulți dintre cei care aspiră a fi
programatori nu îndeplinesc această premiză. Unii dintre ei au avut dificultăți
nu pentru că nu ar ști să vorbească, ci pentru că nu au realizat intensitatea
cu care se pune accent pe acest aspect.
%TODO carte vs. curs
Nu te teme, prima fază de tutelare\footnote{În cadrul programului de tutelare
descris pe {\phpro}} are exact acest scop: să te aducă pe linia de plutire.

Și nu în ultimul rând, este nevoie de buna stăpânire a limbii engleze. Fără ea,
un viitor programator oricum nu ar avea succes -- acesta fiind deseori
confruntat cu necesitatea de a citi documentații în engleză, ba mai mult, de
a își documenta aplicațiile în engleză. Apropo de documentație, toate proiectele
de succes au o documentație bună în engleză. Nu te impacienta dacă nu te simți
pregătit să scrii ceva în engleză, până la capitolul 3 inclusiv vei avea șansa
de a-ți îmbunătăți engleza doar citind.

% }}}
% {{{ Section: Convenții folosite
\section*{Convenții folosite}
\phantomsection
\addcontentsline{toc}{section}{Convenții folosite}

Pentru a face lectura cât mai plăcută, cartea de față respectă anumite
convenții, atât de natură tipografică, cât și inerente comunității din jurul
ei.

% TODO carte vs. curs
În primul rând, pe prima pagină se află un \textit{link} către pagina de
start a proiectului. Această pagină va fi numită mereu \textit{pagina {\phpro}}.
Următoarea secțiune descrie despre ce este vorba în detaliu.

În al doilea rând, când este menționată wikipedia, textul se referă la
versiunea originală în engleză a site-ului, \url{http://en.wikipedia.org/}. În
carte nu se va face referire la niciun \textit{site} în limba română.

Expresiile ``pagina PHP'' și ``manualul PHP'' se referă la paginile oficiale
\url{http://php.net/} și respectiv \url{http://www.php.net/manual/en/}. Demn de
menționat este că se vor folosi doar \textit{site}-urile oficiale ale
produselor menționate.

\attention{Urmarea link-urilor, în special cele către wikipedia și către
manualul PHP, \textit{nu} este opțională. Informațiile prezentate în acele
pagini \textit{fac parte} din cartea de față.}

Cuvintele importante sunt scrise \textit{cursiv}, termenii și noțiunile
importante sunt scrise \textsl{înclinat}, iar cuvintele care fac referire la
nume de funcții, comenzi sau instrucțiuni care trebuie introduse într-un fișier
sau ca comandă, sau cuvinte cheie specifice unui anumit limbaj sunt scrise cu
font
\texttt{neproporțional}%
\footnote{\url{http://en.wikipedia.org/wiki/Monospaced_font}}.

Apăsările de taste sunt scrise în chenare, astfel \keystroke{ENTER} înseamnă
apăsarea o dată a tastei enter, iar \keystroke{CTRL + F} înseamnă apăsarea tastei
\keystroke{CTRL}, și în timp ce aceasta este ținută apăsată, apăsarea
adițională a tastei \keystroke{F}.

Codul sursă, de obicei în PHP, va arăta în felul următor:

\lstinputlisting[caption={Convenție listare}]{cap00/1-listing-convention.php}

Numerele de pe marginea stângă reprezintă numerele liniilor de cod
corespunzătoare și nu trebuiesc scrise. Ele deservesc unei mai ușoare
identificări a instrucțiunilor din cod în explicațiile din text.

Scopul cititorului ar trebui să fie însușirea noțiunilor atât de bine, încât
să fie capabil să producă singur coduri sursă precum cele prezentate aici. Din
acest motiv, codurile sursă nu ar trebui copiate, ci scrise de mână, de la zero,
în mod ideal din memorie, cu cartea închisă.

%TODO carte vs curs
În particular, cursantul din cadrul programului de tutelare {\phpro} nu ar
trebui să aibă ca scop simpla rezolvare a exercițiilor, ci întelegerea
noțiunilor necesare pentru rezolvare ar trebui să fie scopul său.

Atunci când sunt introduși termeni noi, se oferă și traducerea lor în engleză,
în paranteză.  Exemplu: \begin{quote} Un astfel de atac se numește \textsl{man
in the middle}, deoarece atacatorul se află la mijloc, între cele două
capete (en.  \textsl{endpoints}).  \end{quote} iar la sfârșitul cărții se poate
găsi o referință a tuturor acestor termeni.

Există trei marcaje diferite pentru natura unor paragrafe:

\attention{Paragrafele care atenționează asupra unui lucru important sunt
marcate ca atare, ca cel de față.}

\good{În unele locuri se atrage atenția asupra unei practici de programare
bune.}

\bad{În același timp, câteodată se atrage atenția asupra unor lucruri care,
deși sunt posibile, nu ar trebui făcute.}

Exercițiile sunt marcate cu un creion, iar numărul de steluțe reprezintă
dificultatea lor, între 0 (nicio steluță) și 3 (trei steluțe). Exemplu:

\begin{Exercise*}[title={Exercițiu de dificultate 1},difficulty=1]

Eu sunt enunțul unui exercițiu de dificultate 1.

\end{Exercise*}

Exercițiile de dificultate zero necesită doar înțelegerea exemplelor și
explicațiilor imediat anterioare și mici modificări sau adăugiri.  Cu cât
solicitarea inteligenței și a capacității de sinteză a cursantului crește, cu
atât crește și numărul steluțelor.

%TODO issue #54
La evaluarea dificultății exercițiilor se procedează în felul următor: în primul
rând, se pleacă de la premiza că cititorul știe toate noțiunile din capitolul
anterior, chiar dacă nu le-a sintetizat pe toate. În același timp, se pleacă de
la premiza că restul capitolelor trecute au fost bine sintetizate.

\attention{Dacă este necesar saltul înapoi cu mai mult de un capitol pentru
a revizui ceva, atunci este un indiciu că nu au fost parcurse toate stadiile de
studiu în mod consecvent. Secțiunea următoare va explica care sunt aceste
stadii.}

%TODO carte vs. curs
Dacă observi încălcări ale acestor convenții, ești rugat să le raportezi pe
pagina de greșeli a {\phpro}.

% }}}
% {{{ Section: Cum să înveți eficient programare
\section*{Cum să înveți eficient programare}
\phantomsection
\addcontentsline{toc}{section}{Cum să înveți eficient programare}

%TODO at chapter 6: remove this
Momentan cartea de față nu acoperă încă materia așa cum își dorește autorul --
nu este completă.  Însă subiectele abordate sunt acoperite complet, cel puțin
la nivel conceptual.

%TODO carte vs. curs
Cartea în sine nu este gândită pentru a fi folosită singură, ci în paralel cu
cursul gratuit oferit de comunitatea {\phpro}. În particular, unele exerciții
chiar nu sunt gândite pentru a fi rezolvate de cititor singur, ci cu susținerea
tutorilor de pe {\phpro}.

%TODO carte vs. curs
Pe {\phpro} găsești și ajutor sub formă de idei și indicii pentru rezolvarea
exercițiilor. Întreabă-i pe ceilalți cursanți sau pe tutori, cel mai probabil
cineva știe răspunsul.

Dacă setul de scule de programare folosite este lancea de programator, atunci
terminologia este vârful lancei.  Care este diferența dintre un toiag tocit, și
o lance fără vârf?  Exact, nici una. Nu încerca să foloșesti termeni pe care
nu-i înțelegi, ci documentează-te înainte. Cu o lance ascuțită:

\begin{itemize}

    \item te vei putea înțelege mai ușor cu alți programatori; tu îi vei
    înțelege pe ei, și ei pe tine

    \item pe măsură ce termenii înțeleși de tine devin mai complecși, vei putea
    acumula cunoștințe din ce în ce mai complexe bazate pe cele anterioare, în
    ritm exponential. La început ți se va pare frustrant, însă dacă vrei să
    devii bun, oricum va trebui să înveți termenii odată și-odată. Deci de ce
    să nu faci totul ca la carte de la bun început?

    \item un programator profesionist știe mai mult de un singur limbaj de
    programare; ai fi uimit dacă ai afla câți termeni și câte concepte sunt
    comune multor limbaje. Dacă știi terminologia, chiar dacă ai învățat-o în
    (cu) PHP, vei putea trece la un nou limbaj cu mult mai puține eforturi.
    Primul limbaj (învățat corect) este cel mai greu, apoi ți se va pare floare
    la ureche

\end{itemize}

\good{Cititorul trebuie să învețe \textit{terminologia}, să o înțeleagă și să
o folosească încă din momentul introducerii sale.}

Cititorul trebuie să urmeze
\textsl{link}-urile\footnote{\url{http://en.wikipedia.org/wiki/Hyperlink}}
    \index{en-ro}{link}%
    \index{important}{link}%
în timp ce
studiază; acestă carte nu este și nu va fi niciodată ``completă'' -- se pleacă
de la premiza că cititorul citește și înțelege ce se află la acele link-uri
\textit{înainte} de a trece mai departe.

%TODO: asta nu ar trebui sa se intample: sa fie introdusi prematur termeni
Notele de subsol sunt importante; dacă acestea introduc termeni neexplicați
anterior sau în imediata vecinătate, atunci acestea trebuiesc reținute și
făcute legături atunci când termenii respectivi sunt introduși pentru prima
oară.

Se pleacă de la premiza că cititorul are un anumit nivel de inteligență. Asta
nu înseamnă că nu sunt luate în serios orice nelămuriri. Însă este de așteptat
ca noțiunile prezentate să fie citite cel puțin, și apoi înțelese. Nu are rost
să citești o carte dacă ... nu o citești cu trup și suflet. Atunci când se
lovește de o problemă de înțelegere, cititorul trebuie să o ia gradual,
netrecând la următorul stadiu până nu îl îndeplinește pe cel anterior.

\attention{Stadiile{\footnotemark} sunt: citire, înțelegere, sinteză, imaginație
(jonglarea cu noțiunile), inovație.}

\footnotetext{Noțiunea de \textit{stadiu de învățare} este extinderea autorului
a sistemului japonez \textsl{shu-ha-ri}
    \index{term}{shu-ha-ri}%
(en. \textit{retain-detach-transcend},
    \index{term}{retain-detach-transcend}%
jp. \jptext{守 破 離}). Detalii pe
\url{http://www.makigami.info/cms/japanese-learning-system-japan-36}.}

A sintetiza înseamnă a face legături cu toate celelalte noțiuni deja învățate.
De exemplu vei învăța ce înseamnă un \textit{array}, iar peste câteva capitole
vei face cunoștință cu \textit{obiecte}. Dacă vei sintetiza cum trebuie, îți vei
da seama singur că este foarte posibil să ai un \textit{array} de
\textit{obiecte}.

A jongla cu noțiunile are ca efect practic faptul că cititorul știe să pună în
practică și să combine lucrurile învățate de ca și cum acele noțiuni ar fi fost
inventate de el.

\attention{Îți poți ușura procesul de sinteză asimilând terminologia
încă din momentul introducerii ei.}

%TODO dumb example
Această sinteză e foarte importantă, și de fapt, fiecare dintre noi o face de
când era copil. De exemplu, ai văzut-o pe mama ta tăind legumele cu cuțitul.
Mai târziu, la joacă, ai avut nevoie să tai o ață, și nu aveai decât un cuțit
în apropiere. Ți-ai dat seama că poți tăia ața cu cuțitul, deși nu este
o legumă. Altfel spus, ai sintetizat scopul uneltei ``cuțit'': să taie ceva.

Lucrarea de față explică foarte bine noțiunile, de la zero, însă sinteza
acestora îi este lăsată cititorului. Motivația de a proceda așa este de natură
pragmatică: după cum sugerează subtitlul cărții -- \textit{\thesubtitle} --
scopul este calea către profesionalism, iar după cum știm deja, aptitudinea de
\textit{a sintetiza} noțiuni este una importantă pentru un viitor programator.
În plus, autorul este un darwinist convins, și dacă cititorul nu reușește să
devină independent pe plan profesional în programare (web sau altceva), atunci
poate că este mai bine așa. Ultimul lucru pe care ni-l dorim sunt cursanți care
intră pe o ramură a pieții muncii în care nu ar avea succes.

%TODO uncomment at chap 6
Capacitatea de sinteză pe care cititorul o va fi având la sfârșitul cărții mai
are încă un efect pozitiv asupra viitorului profesionist: în programare, acesta
va fi confruntat cu nevoia de a reutiliza codul pe care-l scrie, astfel încât
să nu fie nevoit să rescrie același cod iar și iar, doar pentru că trebuie să-l
personalizeze puțin. Însă pentru a putea face codul atât de flexibil încât să-l
poată adapta cu ușurință, acesta trebuie să prevadă cazuri ``imprevizibile'';
altfel spus, scopul este ca cititorul să se gândească la imposibil.

\good{Nu este recomandată copierea pur și simplă a exemplelor de cod din carte,
deoarece există riscul de a apare ulterior incapacitatea de a scrie ceva de
unul singur. În schimb, este recomandată citirea cu atenție a codului și
explicațiilor din jurul său, apoi \textit{închiderea cărții} și rescrierea
codului din minte, împreună cu argumentarea (pe baza explicațiilor citite) de
ce un lucru e făcut într-un anumit fel, sau chiar de ce e necesar în primul
rând.}

Este într-adevăr mai ușor să copiezi cod, însă cursantul va fi pus în situația
de a scrie cod care nu mai există altundeva. Deci este recomandată scrierea
individuală a scripturilor, de la zero, și de ce nu, comiterea de greșeli.
Atunci când viitorul programator face o greșeală și PHP emite o avertizare sau
eroare, acel mesaj de eroare trebuie citit și înțeles, apoi codul trebuie
corectat. Astfel, în viitor, programatorul va fi mult mai fluent în rezolvarea
rapidă a problemelor, pe măsură ce acestea apar.

Copierea este mai ușoară în primă instanță, dar la un moment dat, va fi necesară
inventarea autonomă a unui script. Deci antrenarea de la început a scrierii de
cod sursă din minte, și de ce nu, a face greșeli, sunt bine venite. Atunci când
faci o greșeală și PHP îți spune asta, citește cu atenție mesajul de eroare,
apoi corectează codul, și reține pentru fiecare fel de greșeală ce eroare
generează, pentru ca în viitor să poți identifica mai rapid greșelile pe baza
mesajelor de eroare pe care ți le arată PHP.

%TODO care putere de imaginație? Paragraful nu are context.
\attention{Această \textit{putere de imaginație}, în combinație
cu \textit{capacitatea ta de analiză și sinteză}, și pe o fundație solidă
a \textit{înțelegerii conceptelor și termenilor} cu care intri în contact,
sunt cheia succesului garantat.}

\subsection*{Comunitatea}
\phantomsection
\addcontentsline{toc}{subsection}{Comunitatea}

\textit{{\thetitle} -- {\thesubtitle}} nu este pur și simplu o carte, ci
o comunitate și o serie de servicii pe care această comunitate le oferă. Cartea
de față constituie doar scheletul, fundația studiului. Pentru a beneficia deci
de aceste servicii, cititorul cărții trebuie fi și cursant în cadrul
comunității.

Pagina {\phpro} este pagina de start a comunității. Printre serviciile oferite
se numără:

\begin{itemize}

    \item verificarea soluțiilor exercițiilor și oferirea de indicii acolo unde
        cursantul s-a blocat, individual, pentru fiecare cursant în parte, exact
        acolo unde are nevoie

    \item clarificarea nelămuririlor pe care cursantul le are în urma citirii
        explicațiilor

    \item articole care întregesc conceptele prezentate în carte; excursuri

    \item garanția că cursanții\footnote{În special cei care au reușit să ofere
        soluții la primele trei exerciții din capitolul 2, eventual cu
    susținerea tutorilor} au într-adevăr potențialul de a deveni profesioniști

    \item servicii care sunt folosite în viața reală a unui programator

\end{itemize}

Comunitatea {\phpro} nu este un loc unde se poate primi ajutor la problemele de
care te-ai lovit pe cont propriu. Altfel spus, comunitatea noastră este strict
una de studiu.

\subsection*{Exercițiile}
\phantomsection
\addcontentsline{toc}{subsection}{Exercițiile}

Exercițiile sunt parte integrantă a studiului. Scopul exercițiilor nu este numai
de a te testa, ci și de a te învăța lucruri noi. De fapt, unele exerciții au
menirea exclusivă de a te învăța ceva.

Indiferent de scopul fiecărui exercițiu, poți apela la comunitatea {\phpro}
pentru susținere, sfaturi și indicii la exerciții. În fapt, chiar va trebui să
o faci la unele exerciții -- vei avea nevoie de asta -- de exemplu la
exercițiile la care ți se va cere să programezi un proiect într-o echipă,
împreună cu alți programatori.

Desprinzăndu-te de comunitatea {\phpro}, riști să studiezi ceva de unul singur
și să ai impresia că ai înțeles totul corect, însă lucrurile învățate se pot
așterne greșit în mintea ta, și la un moment dat te vei lovi tu însuți de
probleme din cauza asta.

Având însă permanent, la fiecare exercițiu, un tutore lângă tine care te
îndrumă, șansele ca un concept de programare să fie înțeles și aplicat
greșit scad considerabil.

Unele exerciții vor fi direct legate de comunitate și de serviciile pe care
aceasta le oferă. În capitolul patru de exemplu, exercițiile îți vor cere să
formezi echipe cu alți cursanți, folosind scule de programare așa cum sunt
folosite în viața reală a unui programator, precum un \textsl{bug tracker} sau
un \textsl{revision control system}.

Însă pentru a primi acces la aceste servicii pe care comunitatea {\phpro} le
oferă gratis, trebuie să rezolvi toate exercițiile anterioare sub tutela
comunității, dovedind astfel că ai potențialul unui programator bun.

% }}}
% {{{ Section: Cum pot ajuta?
\section*{Cum pot ajuta?}
\phantomsection
\addcontentsline{toc}{section}{Cum pot ajuta?}

Atât programatorii experimentați, cât și începătorii, pot ajuta, iar ajutorul
lor este apreciat în egală măsură.

Punctul de întâlnire pentru toți este {\phpro}, unde poate fi găsită îndrumarea
necesară prin diferite canale de comunicare, IRC sau e-mail.

De la cititorii avansați, ne bucură atenționarea asupra greșelilor ce pot exista
în carte. Acest lucru se poate face prin \textit{bug tracker}-ul proiectului.
Corecturile directe, prin sistemul de versionare, sunt și ele bine venite.

Programatorii cu experiență sunt bine veniți să urmeze \textsl{ghidul de
tutelare} pe care îl punem la dispoziție. Este totuși demn de menționat faptul
că tutelarea e diferită de simpla programare a unui proiect -- atunci când
îndrumi pe cineva, metodologia și psihologia joacă un rol important. În
consecință, doar pentru că știi programare, nu înseamnă că nu vei avea nevoie de
o perioadă de acomodare, în care să vezi cum funcționează sistemul. Ceea ce
putem spune este că această muncă necesită atenție la detalii, multă dedicație,
și timp.

În privința calității cărții, există trei mari probleme:
\begin{itemize}

\item Nu cred în cacofonii. Consider că propria imaginație e singura vinovată
    dacă ``vezi'' alte lucruri când citești. Ca atare, refuz să le corectez. Ba
    mai rău, corectarea lor prin folosirea virgulei sau reformulări mai mult ar
    îngreuna inteligibilitatea textului.

\item Sunt folosite xenisme. Resursele în limba engleză sunt cele mai acurate și
    cele mai actuale, din acest motiv nu este evitată folosirea lor. Acest lucru
    îți va permite, pe \textit{termen lung}, să te poți ajuta singur. Pentru
    a articula un xenism este pusă o cratimă, și apoi particula specifică. De
    exemplu \textsl{web}-ul; însă: \textit{Internetul} -- deoarece cuvântul
    internet există în limba română.

\item Este foarte posibil să întâlnești formulări ciudate, cu ordinea cuvintelor
    inversată, și altfel de greșeli similare. Cauza acestui lucru este că 90\%
    din timp vorbesc germana, lucru care-și lasă amprenta.  Corecturile sunt
    binevenite.

\end{itemize}

% }}}
% {{{ Section: O privire de ansamblu a capitolelor
\section*{O privire de ansamblu a capitolelor}
\phantomsection
\addcontentsline{toc}{section}{O privire de ansamblu a capitolelor}
%TODO aici vine roadmap-ul

\begin{enumerate}

\item Rețelistică -- noțiunile de rețelistică sunt necesare pentru a înțelege
    mai ușor apoi lucruri legate de securitate, optimizare sau servicii web

\item Controlul fluxului de execuție și de date -- prezintă constructele pentru
    controlul informațiilor în cadrul aplicației

\item Reutilizarea și modularizarea codului -- împărțirea codului în funcții și
    fișiere, separarea logicii aplicației de vizualizare

\item Baze de date și lucrul în echipă -- cum să lucrezi în echipă, de la
    anumite reguli de comunicare, până la planificarea timpului și git;
    documentarea proiectului; debugging și profiling pentru a lua cele mai bune
    decizii; database design până la și inclusiv 3rd normal form

\item Securitatea aplicațiilor web -- XSS, sql injection, CSRF

\item Programare orientată pe obiecte -- OOP, concepte generale, câteva patterns
    (helper, strategy, factory, singleton), test-driven development, SPL

\item ajax, json, servicii (REST, SOAP, XML-RPC), XML, PDO și alte delicii, și
    ca ``ultima frontiera'': php internals.

\end{enumerate}

Începând cu capitolul 4, proiectele se vor realiza în echipe, iar tutorii vor
avea doar rolul de consilieri. Proiectele rezultate astfel vor aparține
respectivilor programatori, și vor putea fi folosite pentru portofoliile
cursanților.

% vim: set foldmarker={{{,}}} foldlevel=0 foldmethod=marker:

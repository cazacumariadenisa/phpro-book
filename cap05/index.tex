\chapter{Securitatea aplicațiilor web}
\vskip -25pt
\textit{Intro}
\vskip 5em

% Despre vectori de atac, ce înseamnă securitate, accent pe securitatea din perspectiva programatorului.

\section{Ce înseamnă securitate}
Securitatea este un subiect dificil deoarece nu există o rețetă
universal valabilă pentru a crea ceva sigur.

Dar de ce nu există
o astfel de armă secretă?
Pentru a înțelege asta, trebuie să înțelegem natura conceptului de
\textit{a fi sigur}. 

Pentru noi ca programatori, ne interesează în special securitatea
aplicației. Însă aplicația noastră nu este de sine stătătoare: ea
se află pe un server care are un sistem de operare, care este conectat
la Internet, și deseori la o rețea locală centrului de date în
care se află acest server, iar la acest server unele persoane
din cadrul firmei ce deține centrul de date pot avea acces fizic.

Și cum ceva ori este sigur, ori nu este, 
%TODO finish me
\section{Securitatea la nivelul aplicației}
%TODO write me
verificarea inputului, POST forms don't make your app secure,
security by obscurity

includerea fișierelor, definirea unei constante "guard",
punerea fișierelor în afara \texttt{htdocs/}

XSS și CSRF

%TODO implicațiile de securitate "http is stateless"

%TODO exercițiu: Al doilea exercițiu de hacking
TODO: exerciții de hacking. O infrastructură exactă răm\^ane să fie inventată
pentru scenarii mai complexe. Cel mai probabil cursanții vor primi un
{\glqq}private key{\grqq}, a.\^i. toate atacurile (simulative, desigur, nu reale)
vor putea fi traced back.

TODO: exerciții de securizare a scripturilor trecute.

TODO: ne trebuie o evaluare serioasă a vectorilor de atac dacă oferim
{\glqq}attack gates{\grqq}. Cu siguranță vom verifica IP-ul, și doar către
IP-ul care folosește private key se vor trimite atacurile (i.e. cursantul
va folosi serverul phpro pentru a-și ataca propriul server -- proper NAT has to be
in place on his side).
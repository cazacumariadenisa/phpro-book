\section{O altfel de reîmprospătare}
%\linenumbers

Vreau să stabilesc niște lucruri care probabil nu sunt evidente
pentru tine, sau asupra cărora nu ai insistat prea mult
când ai început să înveți să creezi \textit{site}-uri statice
cu HTML, eventual cu CSS sau poate chiar cu interacțiune
în interiorul browserului cu JavaScript.

HTML este un \engl{limbaj de formatare}{markup language}.
Cu el nu controlezi ceva, nu iei decizii, deci nu este
un limbaj de programare. În HTML doar
structurezi un document. În mod ideal îl structurezi
semantic (\href{http://en.wikipedia.org/wiki/Semantic_HTML}{semantic HTML}),
pentru a putea fi interpretat mai bine de motoarele de căutare web
(en. \textsl{search engine}).

Există multe formate de fișiere, și chiar în capitolul anterior ai
lucrat cu două astfel de formate -- formatul de configurare specific apache, și protocolul
HTTP. Dacă îți amintești,
ți-am explicat pe rând ce înseamnă directive precum \texttt{Include}.
Semnificația acestor directive, sau general spus, semnificația oricărei
entități specifice unui limbaj (fie el de markup -- HTML, de configurare
-- \texttt{httpd.conf}, de programare -- PHP, sau un protocol de
comunicare -- HTTP) constituie \textsl{semantica}
acelui limbaj.

Semantica unui anumit construct al unui limbaj (de orice natură
ar fi el) este strâns legată de \textit{contextul} în care se află acel
construct.
De exemplu, pentru a cere o resursă cu HTTP, am văzut că
poți începe cererea cu GET, urmat
de o cale absolută (care începe cu {\glqq}/{\grqq}) din cadrul \texttt{Host}-ului
în cauză. Calea respectivă are semnificația pe care o vrem
doar în contextul lui {\glqq}GET{\grqq}, ca dovadă că atunci când vedem
căi de genul \texttt{/script.php} pe site-uri web,
acestea nu sunt interpretate automat ca parametri GET, și
deci browserul nostru nu este redirecționat către acele pagini.

\textsl{Contextul} a ceva înseamnă în ce punem acel ceva pentru a
avea o anumită semantică. De exemplu, sarea pusă în contextul
gătitului are semantica de \textit{condiment}, dar dacă
o pui pe o rană, are semantica de \textit{ceva care provoacă durere}.
Cu alte cuvinte, \textsl{context} înseamnă
\textsl{circumstanțe} sau \textsl{mediul înconjurător}. 

Analog, {\glqq}HTTP/1.1{\grqq} are semantica de {\glqq}protocolul și versiunea folosită{\grqq}
în contextul \textit{request line}-ului HTTP al unui \textit{request HTTP}.

\begin{Exercise}[title={Întrebări de sinteză},difficulty=2]
Însă și GET însuși are semnificația pe care ai întâlnit-o în capitolul
trecut doar într-un anumit context semantic.

\Question Care este acest context semantic?
\Question În ce context semantic are semantica întâlnită constructul \texttt{Include}?
\Question În ce context semantic are sens comunicarea în limbajul (în protocolul) HTTP?
\ExeText Determină contextul semantic în care următoarele constructe au semantica pe
care o intuiești ca cunoscător al limbajului HTML:
\Question <td>
\Question <tr>
\Question <body>
\Question <html>
\Question href
\end{Exercise}


\attention{Dacă ai avut dificultăți majore la
răspunderea întrebărilor de sinteză, te rog ia atitudine. În primul
rând, plec de la premiza că citești cu atenție, și că reții tot
ce-ți povestesc. Toate noțiunile pe care le introduc, le introduc
pentru că astfel voi putea explica lucruri destul de complicate
mai târziu, pe baza celor spuse aici. \textit{Asta îți va permite să
știi multe învățând cât mai puțin}. {\glqq}Dezavantajul{\grqq} este că
va trebui să fii concentrat la ce citești, și să sintetizezi
singur mult. Sinteza aceasta este un exercițiu perfect pentru
tine ca viitor programator, deoarece atunci când vei programa
vei fi confruntat cu această nevoie de a sintetiza lucruri.
Metoda mea de predare, deși dură, te pregătește foarte bine pentru
cariera ta de programator. Deci dacă simți că nu ești stăpân
pe ce ai învățat până acum despre rețelistică
și despre semantică, recitește acum, până nu te pierzi definitiv.
Când recitești, urmează link-urile menționate -- după cum am spus
în capitolul Introducere, acestea nu sunt lectură opțională.}

Pe lângă semantică, un limbaj mai are și o \textsl{sintaxă}. Regulile sintactice
ale limbajelor sunt necesare pentru a crea contextul semantic în
care vor exista constructele acelui limbaj.

Vreau să ilustrez asta cu un exemplu: în protocolul HTTP, GET trebuie
să fie separat de cale printr-un spațiu. Asta este o regulă sintactică
a limbajului, standardizată prin
RFC-uri,
însă practic separatorul ar putea fi orice altceva.
Însă un separator trebuie să fie acolo, altfel calculatorul (browser-ul
sau daemon-ul) nu ar putea decide unde începe cuvântul cheie {\glqq}GET{\grqq}, unde
se termină, și unde începe calea către resursa pe care o dorim.
Aceste reguli sintactice
permit programelor precum daemon-uri și clienți HTTP
să parseze (să {\glqq}înțeleagă{\grqq}) datele comunicate reciproc.

Există multe posibilități de a exprima sintaxa unui limbaj, însă acestea
sunt mult prea complexe pentru noi. Însă există un standard nescris
pentru a specifica sintaxa unor constructe simple, într-o singură linie.
Ea se leagă de necesitatea unui anumit parametru.

De exemplu, sintaxa constructului GET ar putea fi:\\
\begin{verbatim}'GET ' <RESOURCE> ' HTTP/1.1'\end{verbatim}
\texttt{RESOURCE} este pus între < și >, ceea ce denotă că
este un parametru necesar, care trebuie specificat.
'GET ' (inclusiv spațiul) și ' HTTP/1.1' sunt puse
între apostrofuri pentru a arăta că sunt lucruri
ce trebuie scrise exact așa cum sunt. \texttt{RESOURCE}
este numele simbolic al parametrului, pe care îl putem
refolosi în documentația limbajului (în cazul nostru,
documentarea limbajului HTTP, sau mai bine spus, a unei
cereri HTTP de bază).

Pentru a specifica că un parametru e opțional, îl punem între [ și ].
Astfel, sintaxa unei cereri HTTP ca cea pe care am făcut-o în capitolul
anterior, împreună cu descrierea ei, ar putea arăta astfel:
\begin{verbatim}
'GET ' <resource> ' HTTP/' <version> <enter>
['Host: ' <name enter> ]<enter>

	- resource	= the absolute path to the resource
	- version 	= the version of the HTTP protocol used;
             currently only 1.0 and 1.1 are supported
	- name	    = the hostname
	- enter    = press return once
\end{verbatim}

Pe lângă lucrurile evidente pe care ni le spune această specificație
sintactică, ne mai spune și un lucru care probabil ți-a scăpat:
câmpul \texttt{Host} este opțional, dar dacă îl specificăm, atunci
trebuie să specificăm și parametrul \texttt{name}, și să și apăsăm o dată enter.

Cu siguranță ai realizat că astfel de reguli pot fi incluse una în alta,
creând reguli destul de complexe.

Un limbaj precum cel de mai sus, care folosește <,>,[,] pentru a specifica
un alt limbaj se numește un \textsl{metalimbaj}. \textit{Meta} înseamnă
\textit{care descrie} - \textit{un limbaj care descrie limbajul}.



Cel mai probabil ai întâlnit deja \textsl{tag}-ul HTML \texttt{<meta>}, de aici
îi provine numele. Diferența este că \texttt{<meta>} nu se referă la limbaj (HTML
în cazul de față), ci la informații. Ceea ce \texttt{<meta>} ne spune despre
documentul HTML în care se află se numesc \textsl{metainformații} - \textit{informații
care descriu informațiile din documentul curent}.

Probabil ai folosit deja un forum pe web. Probabil că acel forum
îți spunea la un moment dat că autorul unei intrări (al unui \textsl{topic}
sau \textsl{thread}) se numește Xulescu. Ei bine, în timp ce
intrarea propriu-zisă constituie informația, numele autorului
este o \engl{metainformație}{metadata} - o informație despre
informație.
%Vom reveni mai târziu asupra conceptului de metainformație.

\attention{
La ce îți folosește această cunoaștere despre metadate, metalimbaje, sintaxă
și semantică? În primul rând, ai învățat primul cel mai important
lucru pragmatic din viitoarea ta carieră de programator: să citești manualul
(PHP sau orice altceva) -- chiar dacă încă
nu ești conștient de asta.

\smallskip

În al doilea rând, meta-ceva-urile te vor însoți
în toate aplicațiile pe care le vei programa. Uită-te
la toate aplicațiile pe care le folosești, vei vedea
că toate au niște metadate. Singurul lucru care
te-ar împiedica să vezi asta este că datele și
metadatele se întretaie atât de mult încât e
greu să le identifici pe fiecare.

\smallskip

Și în al treilea rând, pentru a te pregăti pentru
exercițiile următoare care au menirea de a te dezgheța la minte puțin,
deoarece, din păcate, nu știu nimic despre cititorul meu, însă trebuie
să mă asigur cumva că este pe aceeași lungime de undă ca mine, ceea ce-i
permite să asimileze cât mai eficient cunoașterea prezentată
în continuare, fără risipă de cuvinte.}

%TODO spune că e un pattern, că 'aba' este potrivit de X Y X, deoarece respectă 'modelul'


\begin{Exercise}[title={Reguli sintactice},difficulty=1]
Fie regula sintactică
%TODO corectează 'A' 'B' 'C'
\begin{verbatim}
                 [A] A [A A] [A [A <B>]] <C>
\end{verbatim}
\ExePart
Care dintre următoarele inputuri o respectă?
\Question AABC % nu
\Question AAAAC % da
\Question AC % da
\Question AAAC % da
\Question AAABC % da
\ExePart
Detașează-te de nivelul abstract al acestei reguli sintactice, și fă o afirmație
\textit{pragmatică} despre B. Afirmația începe așa:
\textit{Într-un input valid, B este mereu \ldots} % este mereu precedat de cel putin 3, de maxim 6 A-uri
\end{Exercise}

Metalimbajul prezentat nu este bătut în cuie. În primul rând,
caracterele speciale ale limbajului <,>,[,] și ' pot fi schimbate
în orice, atâta timp cât documentezi aceste schimbări aduse de tine.

Deasemenea, nu este decât un standard nescris, și îl poți
extinde în ce fel ai nevoie. Din nou, important este doar
să documentezi {\glqq}extensiile{\grqq} aduse metalimbajului astfel
încât ceilalți programatori să îți înțeleagă specificația
limbajului pe care îl descrii cu ajutorul acelei
extensii proprii a metalimbajului.

De exemplu, să zicem că vrem să introducem un nou
construct în acest metalimbaj care să însemne \textit{simbolul
din stânga mea poate apărea o dată sau de mai multe ori}, și
ne decidem să folosim simbolul '+' pentru asta.

Astfel o regulă de genul:
\begin{verbatim}
	<FOO+> [BAR+]
\end{verbatim}
S-ar putea citi ca: \textit{Unul sau mai mulți \texttt{FOO} urmat de zero sau mai
mulți \texttt{BAR}}. Un astfel de simbol precum '+' se numește \textsl{cuantificator}.
Ar fi la îndemână să stabilim, în documentația extensiei noastre adusă metalimbajului,
că '+' poate cuantifica orice entitate din stânga sa, inclusiv o grupare <> sau [].

Documentația ar suna așa:
\begin{verbatim}
+ = repeat the entity on its left once or multiple times (a quantifier)
    the entity can be any SYMBOL, <required parameter> or
	[optional parameter]
\end{verbatim}
Iar exemplul de mai sus, în care vrem ca \texttt{FOO} și \texttt{BAR}
să fie separați de un eventual spațiu, ar deveni:
\begin{verbatim}
	<FOO ' '>+ [BAR ' ']+
\end{verbatim}

Până acum definițiile noastre sintactice se limitau doar la
o singură regulă (o singură linie), însă am putea introduce
constructe în metalimbaj care ne-ar da voie să atribuim nume
acestor reguli, și să refolosim acele nume în definițiile altor
reguli sintactice, creând astfel interdependențe între reguli,
și deci crea specificațiile unor limbaje foarte complexe.

\attention{Majoritatea limbajelor de programare, inclusiv
PHP, sunt definite în astfel de limbaje.}

\begin{Exercise}[title={Sintaxa HTML},label={ex:sintaxa_html},difficulty=3]
Acum hai să aplicăm ce am învățat asupra limbajului HTML. În HTML avem
tag-uri (ex. \texttt{<html>}) care au atribute și valori.

\Question Crează o specificație sintactică a limbajului HTML folosind
doar cuvintele cheie TAG, ATTRIBUTE și VALUE, care combinate
cu [] și <> să reflecte cât mai corect sintaxa limbajului.

Specificația trebuie să valideze orice text HTML valid.
Un exemplu de input ar fi:
\begin{verbatim}
<form method="get">
   <checkbox name="hello" checked>
   <input type="submit">
</form>
\end{verbatim}
\ExeText
Note:
\begin{itemize}
\item Regula creată \textit{nu} trebuie să ia în calcul sintaxa
specifică XHTML (în care de exemplu '<img>' ar fi greșit, doar '<img />' este valid).
\item Se pleacă de la premiza că inputul este cel mai curat HTML posibil, că
sunt folosite " pentru a delimita valorile atributelor (dacă acestea există), că
nu e mai mult de un spațiu acolo unde e nevoie de spațiu, ș.a.m.d. Pe scurt:
folosește-ți intuiția pentru a decide ce înseamnă {\glqq}cel mai curat HTML posibil{\grqq}.
\item Deoarece caracterele < și > au o semnificație specială în limbajul HTML, pe
care încerci să-l descrii sintactic folosind printre altele
și caracterele < și > înseși, va trebui să le pui între apostrofuri, pentru a face
diferența între <,> care ne spun în metalimbajul nostru
că acel parametru este necesar, și '<' sau '>' care
ne spun că ne referim la caracterul '<' respectiv '>' în limbajul pe care
vrem să-l descriem (adică HTML însuși).
\item Va trebui să extinzi metalimbajul (nu uita să și documentezi extensiile aduse)
pentru a ajunge la o rezolvare cât mai corectă și completă
\item Exercițiul este destul de dificil. Încearcă să te apropii cât mai mult de soluția
cea mai corectă și completă.
\end{itemize}
\end{Exercise}

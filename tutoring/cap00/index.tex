\introduction{Introducere}

\begin{chapsummary}
Acest document are ca scop clarificarea aspectelor de natură pedagogică și
tehnică din spatele cursului și al exercițiilor. Cele scrise aici se bazează
exclusiv pe experiențele culese de la începutul cursului.

Documentul este destinat tuturor celor care doresc să îndrume la rândul lor.
\end{chapsummary}

TO be done

\section{Intrarea în curs}

\section{Reintrarea în curs, după o pauză lungă}

Nu există noțiunea de ``ieșire din curs'', dar există lucruri pe care un cursant
în hibernare trebuie să arate că le știe atunci când vrea să continue de unde a
rămas:

\begin{itemize}
\item unde a rămas, la ce stadiu
\item cum funcționează cursul dpv organizatoric
\item cât de bine stăpânește cele studiate -- se stabilește prin discuții repetate,
    iterative, libere, în care cursantul demonstrează că nu și-a pierdut cunoștințele tehnice,
    și că soft-skills nu s-au degradat (atenția în analiza lucrurilor, exactitatea
    în exprimare, etc). Nu suntem stricți la acest aspect, dar nici prea delăsători
\end{itemize}
